\documentclass[10pt]{article}
\usepackage[T1]{fontenc}
\usepackage[utf8]{inputenc}%ATTENTION codage en utf8 ! 
\usepackage{fourier} 
\usepackage[scaled=0.875]{helvet}
\renewcommand{\ttdefault}{lmtt}
\usepackage{amsmath,amssymb,makeidx}
\usepackage{fancybox}
\usepackage{graphicx}
\usepackage{tabularx}
\usepackage[normalem]{ulem}
\usepackage{enumitem}
\usepackage{pifont}
\usepackage{textcomp} 
\newcommand{\euro}{\eurologo}
%Tapuscrit : Denis Vergès
%Relecture : François Hache
\usepackage{pst-plot,pst-text,pst-node,pst-tree,pstricks-add}
\newcommand{\R}{\mathbb{R}}
\newcommand{\N}{\mathbb{N}}
\newcommand{\D}{\mathbb{D}}
\newcommand{\Z}{\mathbb{Z}}
\newcommand{\Q}{\mathbb{Q}}
\newcommand{\C}{\mathbb{C}}
\usepackage[left=3.5cm, right=3.5cm, top=3cm, bottom=3cm]{geometry}
\newcommand{\vect}[1]{\overrightarrow{\,\mathstrut#1\,}}
\newcommand{\barre}[1]{\overline{\,\mathstrut#1\,}}
\renewcommand{\theenumi}{\textbf{\arabic{enumi}}}
\renewcommand{\labelenumi}{\textbf{\theenumi.}}
\renewcommand{\theenumii}{\textbf{\alph{enumii}}}
\renewcommand{\labelenumii}{\textbf{\theenumii.}}
\def\Oij{$\left(\text{O}~;~\vect{\imath},~\vect{\jmath}\right)$}
\def\Oijk{$\left(\text{O}~;~\vect{\imath},~\vect{\jmath},~\vect{k}\right)$}
\def\Ouv{$\left(\text{O}~;~\vect{u},~\vect{v}\right)$}
\usepackage{fancyhdr}
\usepackage{hyperref}
\hypersetup{%
pdfauthor = {APMEP},
pdfsubject = {BTS},
pdftitle = {Métropole septembre 2020},
allbordercolors = white,
pdfstartview=FitH}  
\usepackage[french]{babel}
\usepackage[np]{numprint}
\begin{document}
\setlength\parindent{0mm}
\rhead{\textbf{A. P{}. M. E. P{}.}}
\lhead{\small Brevet de technicien supérieur Métropole}
\lfoot{\small{Services informatiques aux organisations\\ épreuve obligatoire}}
\rfoot{\small{septembre 2020}}
\pagestyle{fancy}
\thispagestyle{empty}
\marginpar{\rotatebox{90}{\textbf{A. P{}. M. E. P{}.}}}

\begin{center} {\Large \textbf{\decofourleft~BTS Métropole septembre 2020~\decofourright\\[5pt]Services informatiques aux organisations}}

\vspace{0,25cm}
  
\end{center}

\textbf{Exercice 1 \hfill 5 points}

\medskip

Une entreprise fabrique des manettes pour consoles de jeux vidéo. Elle en propose de deux tailles différentes: grandes ou bien petites; qui sont de couleur soit noires, soit argentées pour chaque taille ; et qui sont sans fil ou bien à brancher pour chaque taille également. 

On introduit les variables booléennes suivantes:

\setlength\parindent{9mm}
\begin{itemize}[label=\textbullet]
\item $g$ signifie que la manette est grande, $\overline{g}$ que la manette est petite;
\item $n$ signifie que la manette est de couleur noire, $\overline{n}$ que la manette est de couleur
argentée ;
\item $b$ signifie que la manette est à brancher, $\overline{b}$ que la manette est sans fil.
\end{itemize}
\setlength\parindent{0mm}

Cette entreprise fournit plusieurs fabricants qui lui en achètent des quantités analogues. 

Après plusieurs mois de vente, l'entreprise constate que les manettes vendues sont de l'un au moins des 4 types suivants:

\setlength\parindent{9mm}
\begin{itemize}[label=\textbullet]
\item les grandes manettes sans fil ;
\item les petites manettes de couleur noires ;
\item les petites manettes de couleur argentées et sans fil ;
\item les petites manettes qui sont à brancher.
\end{itemize}
\setlength\parindent{0mm}

On note $E$ l'expression booléenne correspondant aux types de manettes les plus vendues par l'entreprise.

On admet que $E= g.\barre{b} + \barre{g}.n + \barre{g}. \barre{n}. \barre{b} + \barre{g}. b$.

\medskip

\begin{enumerate}
\item Traduire par une phrase l'expression booléenne $\barre{g} . b$.
\item Représenter $E$ par un tableau de Karnaugh, puis déterminer une forme simplifiée, à
deux termes, de l'expression booléenne $E$.
\item Traduire par une phrase l'expression simplifiée $E$.
\item L'entreprise souhaite réduire sa production en supprimant les types de manettes non
vendues. Lesquelles doit-elle supprimer ? Justifier votre réponse.
\end{enumerate}

\vspace{0,5cm}

\textbf{Exercice 2 \hfill 8 points}

\medskip

Dans le modèle RGB (Red, Green, Blue), datant de 1931, la couleur et l'intensité de la lumière peuvent être représentées par la matrice colonne $C = \begin{pmatrix}R\\G\\B\end{pmatrix}$, où $R$ représente
l'intensité de la composante rouge, $G$ l'intensité de la composante verte et $B$ l'intensité de la composante bleue. 

L'intensité de chaque composante est, dans le système décimal, un entier compris entre $0$ et $255$ : $0$ désigne l'absence de celle-ci et $255$ désigne l'intensité maximale de celle-ci.

\bigskip

\textbf{Partie A} - Codage de couleurs

\medskip

La couleur \og saumon \fg{} est codée par $\begin{pmatrix}248\\142\\85\end{pmatrix}$ en décimal, l'intensité du rouge est donc 248, celle du vert est 142 et celle du bleu est 85. 

Dans certains logiciels comme Photoshop par
exemple, les couleurs sont codées par 3 nombres hexadécimaux à deux chiffres représentant les valeurs de Rouge, Vert et Bleu. En hexadécimal, cette couleur \og saumon \fg{}
est codée $(F8~;~8E~;~55)$ que l'on notera par la matrice $\begin{pmatrix}F8\\8E\\55\end{pmatrix}$.

\medskip

\begin{enumerate}
\item La couleur \og vert tilleul \fg{} est codée en écriture décimale par $\begin{pmatrix}165\\209\\82\end{pmatrix}$.

Déterminer son codage en hexadécimal, on détaillera la démarche pour la valeur $165$.
\item La couleur \og mauve \fg{} est codée en hexadécimal par $\begin{pmatrix}D4\\73\\D4\end{pmatrix}$.

Déterminer son codage en écriture décimale, on détaillera la démarche pour la valeur
$D4$.
\item Combien de couleurs différentes peut-on représenter avec ce mode de
représentation ? Combien de bits utilise ce codage ?
\end{enumerate}

\bigskip

\textbf{Partie B} - De la lumière vers l'œil.

La rétine d'un œil humain est composée de deux types de récepteurs : les cônes et les bâtonnets. Les bâtonnets sont responsables de la vision à faible niveau d'énergie (vision nocturne dite \og scotopique \fg) et vision à niveaux de gris) et ne perçoivent pas les couleurs. 

Ils mesurent l'intensité de la lumière visible. Les cônes sont responsables de la vision diurne colorée. La vision des couleurs n'est pas toutefois directe, elle est envoyée au
cerveau au moyen d'un signal $S = \begin{pmatrix}i\\l\\c\end{pmatrix}$.

\setlength\parindent{9mm}
\begin{itemize}[label=\textbullet]
\item L'intensité $i$ de la lumière est $i= \dfrac{1}{3}(R + G + B)$ ;
\item L'intensité $l$ des ondes longues est $l = R - G$ ;
\item L'intensité $c$ des ondes courtes est $c = B - \dfrac{R + G}{2}$.
\end{itemize}
\setlength\parindent{0mm}

Par exemple, pour la couleur \og vert tilleul \fg{} codée en décimal par $\begin{pmatrix}165\\209\\82\end{pmatrix}$, l'intensité $i$ de la lumière est $i = \dfrac{1}{3}(165 + 209 + 82) = 152$ ; l'intensité $l$ des ondes longues est $l = 165 - 209 = - 44$  et l'intensité $c$ des ondes courtes est $c = 82 - \dfrac{165 + 209}{2} = - 105$.

On note les matrices : $C = \begin{pmatrix}R\\G\\B\end{pmatrix}$ et $M = \begin{pmatrix}\frac{1}{3}&\frac{1}{3}&\frac{1}{3}\\1&-1&0\\-\frac{1}{2}&-\frac{1}{2}&1\end{pmatrix}$.

\medskip

\begin{enumerate}
\item 
	\begin{enumerate}
		\item Donner une égalité reliant les matrices $S$, $C$ et $M$.
		\item Calculer les différentes intensités $i$, $l$ et $c$ du signal lorsque $R = 150$, $G
		 = 90$ et $B = 210$.
	\end{enumerate}
\item Soit $N$ la matrice définie par: $N = \begin{pmatrix}1&\frac{1}{2}&-\frac{1}{3}\\1
&- \frac{1}{2}&-\frac{1}{3}\\1&0&\frac{2}{3}\end{pmatrix}$.
	\begin{enumerate}
		\item Calculer le produit $N \times M$.
		\item Que peut-on en déduire pour les matrices $N$ et $M$ ?
	\end{enumerate}
\item 
	\begin{enumerate}
		\item Prouver que si $M \times  C= S$ alors $C = N \times S$.
		\item Le cerveau reçoit comme signal: $i = 120$ ; $l = 100$ et $c = - 90$.
		
Quelles sont les intensités $R$, $G$ et $B$ de la lumière reçue par l'œil ?
	\end{enumerate} 
\end{enumerate}

\vspace{0,5cm}

\textbf{Exercice 3 \hfill 7 points}

\medskip

En informatique, le code ASCII associe à certains caractères (lettre, chiffre, signe de ponctuation, \ldots) un entier compris entre 0 et 255 que l'on appelle son code ASCII. La fonction code du tableur renvoie le code ASCII du caractère. 

L'extrait de tableur ci-dessous donne le codage de quelques caractères.

\begin{center}
\begin{tabularx}{\linewidth}{|l|*{13}{>{\centering \arraybackslash \footnotesize}X|}}\hline
Lettre 			&A	&B	&C	&D	&E	&F	&G	&H	&I	&J	&K	&L	&M\\ \hline
CodeASCII : $n$	&65&66	&67	&68	&69	&70	&71	&72	&73	&74	&75	&76	&77\\ \hline
$p=f(n)$		&199&206&	&	&	&	&	&	&	&	&	&	&\\ \hline\hline
Lettre			&N	&O	&P	&Q	&R	&S	&T	&U	&V	&W	&X	&Y	&Z\\ \hline 
CodeASCII : $n$&78	&79	&80	&81	&82	&83	&84	&85	&86	&87	&88	&89	&90\\ \hline
$p=f(n)$		&	&	&	&	&	&	&	&	&	&	&	&	&\\ \hline
\end{tabularx}
\end{center}

\medskip

On décide de chiffrer (crypter) une lettre à partir de son code ASCII en utilisant la fonction $f$ définie pour tout entier $n$ tel que $0 \leqslant n < 256$ par : $f(n)$ est le reste de la division euclidienne de $7n$ par $256$, c'est-à-dire que si on note $p = f(n)$, alors $p \equiv 7n \:(\text{mod}~256)$ où $p = f(n)$, avec $p$ entier tel que $0 \leqslant p < 256$.

Par exemple, le code ASCII de la lettre A est 65. On a $7 \times 65 \equiv 199\: (\text{mod}~256)$ et $0 \leqslant  199 < 256$ donc la lettre A est chiffrée par 199.

\bigskip

\textbf{Partie A }- Chiffrement

\medskip

\begin{enumerate}
\item Vérifier que la lettre \og B \fg{} est chiffrée par le nombre $206$.
\item Déterminer le cryptage du mot \og BTS \fg. (On séparera chaque code de lettre par un
espace).
\end{enumerate}

\bigskip

\textbf{Partie B }- Déchiffrement

\medskip

Pour déchiffrer un entier $p$ compris entre $0$ et $255$ (inclus), on calcule le reste de la division euclidienne de $183 \times p$ par $256$ ; autrement dit $n \equiv 183 \times p \:(\text{mod}~256)$ avec $n$ entier tel que $0 \leqslant  n < 256$.

\smallskip

Par exemple, pour $p = 20$, on a $183 \times 20  \equiv 76 \:(\text{mod}~256)$ et donc la valeur chiffrée correspond à la lettre~L.

\medskip

\begin{enumerate}
\item Déterminer la lettre correspondant à la valeur chiffrée $27$. On détaillera la réponse.
\item  Donner le mot de trois lettres correspondant au code chiffré des trois entiers:
234\quad  255 \quad 34.
\end{enumerate}

\bigskip

\textbf{Partie C} - Justification

\medskip

\begin{enumerate}
\item Prouver que $183 \times 7 \equiv 1\:(\text{mod}~256)$.
\item On souhaite justifier comment obtenir l'entier $n$ à partir de l'entier $p = f(n)$.

On rappelle que $f(n) \equiv 7n \: (\text{mod}~256)$.

En déduire que $183 \times f(n) \equiv n \: (\text{mod}~256)$.
\end{enumerate}
\end{document}